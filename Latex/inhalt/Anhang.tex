%! Anhang

\clearpage
\appendix
\clearpage

%! Section Befehl wird umgeschrieben, um Überschriften zu verbergen
%! Kann, falls Überschriften gewollt sind, entfernt oder erst später eingefügt werden.
% Beginn
\makeatletter
\renewcommand{\section}[1]{%
\par\refstepcounter{section}%
\sectionmark{#1}%
\NR@gettitle{#1}%<---------
\addcontentsline{atoc}{section}{\bfseries\protect\numberline{\thesection}{\mdseries#1}}%
\lohead{\textnormal{#1}}%
}
\makeatother
% Ende

%! Anpassung der Darstellung von Abbildungen im Anhang
%! Eine Variante auskommentieren
%? Möglichkeit 1: ohne Nummerierung
%\renewcommand{\bild}[4][1.0]{\begin{figure}[H]
    %\centering
    %\includegraphics[width=#1\columnwidth]{bilder/#2}
    %\caption*{\bfseries Abbildung \mdseries #3}
    %\label{#4}
    %\end{figure}}

%? Möglichkeit 2: mit Nummerierung aber nicht im Abbildungsverzeichnis
\renewcommand{\bild}[4][1.0]{\begin{figure}[H]
    \centering
    \includegraphics[width=#1\columnwidth]{bilder/#2}
    \caption[]{#3}
    \label{#4}
    \end{figure}}

%! Anhang 1
\section{Erster Anhang}
\label{sec:anhang1}
Der erste Anhang der Arbeit.
\clearpage

%! Anhang 2
\section{Inhalt der CD}
CD mit folgenden Inhalten:
\begin{itemize}
    \item dieses Dokument
    \item \LaTeX -Dateien
    \item \href{https://www.youtube.com/watch?v=dQw4w9WgXcQ}{YouTube-Video} als Bonus
\end{itemize}
\clearpage

% Warnungen für vorgefertigte Dokumente deaktivieren
\hbadness=10000
%  Eidestattliche Erklärung und Erklärung zur Prüfung wissenschaftlicher Arbeiten
%! hier zwischen Version für einen oder mehrere Autoren umschalten
\input{vorlage/Erklärung_1Autor}
%\input{vorlage/Erklärung_2Autoren}
%\input{vorlage/Erklärung_3Autoren}
%\input{vorlage/Erklärung_4Autoren}

%  Abstract und Freigabeerklärung zur Bachelorthesis
%! Falls nötig (z.B. bei Diplomarbeit) innerhalb der Datei Wortlaut anpassen, sowie Freigabeerklärung anpassen (erfordert Format Name, Vorname usw.)

%\input{vorlage/Abstract_Freigabeerklärung_Bachelorthesis}

% Warnungen zurücksetzen
\hbadness=1000