\section{Bereich 3}
Auswahl einer Entwicklungsumgebung.
Skizze der Oberfläche der Anwendung (aus Sicht des Endanwenders).
Umsetzung einer ausgewählten Funktionalität in einer der Sprachen Python, C\#, Java oder Swift:

- Visual Studio Code, mit der diversen Plugins
- Docker und Portainer Installation, von SYSM kopieren
- mit Draw.io eine schöne Skizze für die Oberfläche machen, anders als Team Hauschild. (eventuell fix in TailwindCSS das Interface)
- Produkt im Regal anzeigen --> Aus Datenbank Regalplatz für beliebiges Produkt herausfinden --> Anschließend, den Regalplatz markierne, und mittels Depth-First Suchalgorthimus, einen Weg zu diesen Platz im Supermarkt finden --> Python
- Eventuell noch das Frontend in HTML designen, falls uns langweilig ist.


\subsection{Visual Studio Code}

\subsection{Raspberry Pi 4}
\label{sec:raspi}
\subsubsection{Installation und Einrichtung}
Die in der Maske eingebauten Sensoren und der Mikrocontroller erfassen zahlreiche Daten und Informationen, die in diesem Kapitel ausgewertert werden sollen.
Unsere Wahl fiel auf einen Raspberry Pi 4, da dieser uns noch zur Verfügung stand und keine Neuanschaffung notwendig wurde.
Der Raspberry Pi 4 ist ein Einplatinencomputer mit zahlreichen Schnittstellen, z.B. USB, Ethernet, Bluetooth, WiFi und einem Micro-SD-Karten Slot für das \striche{Raspberry Pi OS}-Betriebssystem.
Im Folgenden soll in Kurzfassung beschrieben werden, welche Schritte nötig sind, um mit einem Raspberry Pi Daten von einem ESP32 zu empfangen, diese zu speichern und anschließend auszuwerten.

\begin{enumerate}[leftmargin=*]
\item Steht der Raspberry Pi zur Verfügung und man besitzt eine Micro-SD-Karte, die man verwenden kann, ist es nötig das Betriebssystem herunterzuladen\vglink{https://www.raspberrypi.org/software/operating-systems/}{23.02.2021}.
      Die Entscheidung fiel auf Raspberry Pi OS Lite, da keine grafische Benutzeroberfläche benötigt wird und dadurch Speicherplatz gespart werden kann.
      Das heruntergeladene Image (*.iso - File) kann man nun mit einem beliebigen Programm, in unserem Fall Balena-Etcher\vglink{https://www.balena.io/etcher/}{23.02.2021} auf die SD-Karte schreiben.

\item Im nächsten Schritt werden Änderungen an den Konfigurationsdateien des Raspberry Pi OS' vorgenommen. Dazu gehört das Einrichten einer WLAN-Verbindung und das Aktivieren von SSH, um den Raspberry Pi \striche{headless}, also ohne angeschlossenen Bildschirm \striche{remote} betreiben zu können.
      Die Installation und eine Verbindung per SSH mittels der Software \striche{PuTTY}\vglink{https://www.chiark.greenend.org.uk/~sgtatham/putty/latest.html}{23.02.2021} war erfolgreich.
      Auf dem Raspberry Pi kann nun die Software installiert werden, die für das beschriebene Ziel benötigt wird.

\item Die Container Lösung Docker\onlinezitat{DOCKER2021} bietet einen Ansatz mit sogenannten \striche{Compose}-Dateien und bietet eine hohe Flexibilität der verwendeten Software.
      Um Docker unter Raspberry Pi OS zu installieren führt man folgenden Befehl aus: \mintinline[breaklines]{bash}{curl -sSL https://get.docker.com | sh}.
      Nach Abschluss müssen die Benutzerrechte unter Linux mit \mintinline[breaklines]{bash}{sudo usermod -aG docker pi} angepasst werden.
      Der Standardnutzer \striche{pi@raspberry} darf nun auch Docker-Befehle ausführen.
      Anschließend kann die Docker Installation mittels \mintinline[breaklines]{bash}{docker run hello-world} verifiziert werden.
      Bei erfolgreichem Abschluss erscheint ein \striche{Hello World} auf dem Bildschirm und die Docker Container Runtime ist nun nutzbar.

\item Es besteht die Möglichkeit, einen weiteren Bestandteil von Docker, nämlich Docker-Compose (dazu später mehr) zu installieren.
      Dies ist mit dem folgenden Befehl möglich: \mintinline[breaklines]{bash}{sudo pip3 -v install docker-compose}.
      Als Quelle für diese Installation diente die Docker Dokumentation und dev.to\vglink{https://dev.to/rohansawant/installing-docker-and-docker-compose-on-the-raspberry-pi-in-5-simple-steps-3mgl\#articles-list}{24.02.2021}.
\end{enumerate}

Mit dem Abschluss der Installation von Docker Compose und dem Test der Docker Engine steht einer Installation von verschiedenen Docker Images nichts mehr im Wege.

\subsubsection{Software und Datenübertragung}
\label{sec:compose}
Der Raspberry Pi benötigt zur Datenerfassung eine Datenbank, die Entscheidung fiel auf MySQL, einen Webserver, in diesem Fall \striche{nginx}\vglink{https://www.nginx.com/}{27.02.2021} und eine Verwaltung der Datenbank \striche{phpMyAdmin}\vglink{https://www.phpmyadmin.net/}{27.02.2021}.
Mit den im vorherigen Kapitel erwähnten Compose-Dateien von Docker ist ein solches Setup in wenigen Schritten aufsetzbar.
Beispielhaft ist in Listing \ref{code:compose} die Installation eines \striche{nginx-Containers} gezeigt.
Die vollständige Compose-Datei befindet sich im Anhang \striche{compose.yml}.
Nach dem Abschluss der Installation kann die Datenübertragung an das Raspberry Pi erfolgen.
\begin{code}[H]
      \inputminted[linenos, breaklines, gobble=0, frame=none,
      firstnumber=1,
      firstline=19,
      lastline=25,
      numbers=left,
      numbersep=5pt]{yaml}{code/compose.yml}
      \caption{Compose Datei für einen Webserver-Stack}
      \label{code:compose}
\end{code}

Der gezeigte Code und Ablauf soll beispielhaft für die Übertragung der Entfernung gezeigt werden, die GPS- und Temperaturübertragung erfolgt nach dem gleichen Schema und wird nicht weiter erklärt.

In Vorbereitung der Datenübertragung wurde mittels phpMyAdmin eine Datenbank \striche{sysm} angelegt, die drei Tabellen enthält, eine davon ist die Entfernung.
Im PHP-Code \ref{code:php} wird zunächst in den Zeilen $2$ - $11$ eine Verbindung zur MySQL-Datenbank hergestellt.
Anschließend  wird die aktuelle Uhrzeit und das Datum ermittelt, die Entfernung, die per HTTP-GET Header vom ESP32 übertragen wird, ausgelesen und in die Variable \mintinline{php}{$entfernung} als Integer gespeichert.
Aus dem ganzen wird eine SQL-Query in Zeile $20$ zusammengesetzt und anschließend in Zeile $22$ ausgeführt.
Der Datensatz steht jetzt in der Datenbank und kann zu einem späteren Zeitpunkt (siehe Kapitel \ref{sec:datenauswertung} - \nameref{sec:datenauswertung}) ausgewertet werden.
\begin{code}[H]
      \inputminted[linenos, breaklines, gobble=0, frame=none,
      firstnumber=1,
      firstline=1,
      lastline=28,
      numbers=left,
      numbersep=5pt]{php}{code/enf.php}
      \caption{PHP-Datei zum Schreiben in die Datenbank}
      \label{code:php}
      \end{code}
\newpage