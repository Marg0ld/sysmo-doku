\section{Aufgabenbereich 1}
Beschreibung der Anwendung, Architektur der Anwendung hinsichtlich Hardware und Verteilung der Anwendung.
Dazu gehört auch die Auswahl des mobilen Endgerätes.

\subsection{Beschreibung der Anwendung}
Wer kennt es nicht? Manchmal wird es zu einem wahren Gang durchs Labyrinth einen bestimmten Artikel im Supermarkt zu finden.
Praktisch wäre es eine mobile Anwendung zu entwickeln die genau in diesem Punkt Abhilfe verschaffen kann.
Unser Ziel ist es, eine solche Lösung möglichst benutzerfreundlich und mit geringen Anforderungen an das mobile Endgerät zu entwickeln. \onlinezitat{DOCKER}



\subsection{Aufbau der Anwendung}
\subsubsection{Allgemeiner Aufbau}
Die Anwendung ist wie in Abbildung \ref{fig:sysmo1} zu sehen aufgebaut. 

\bild[1.0]{sysmo1}{Theoretischer Aufbau der Anwendung}{fig:sysmo1}

Bei unserer Art der Umsetzung wird nahezu die vollständige 
Abwicklung und Datenbereitstellung über das Backend betrieben.
Eine ausführliche Beschreibung des Backends findet sich im Kapitel \ref{sec:backend} - \nameref{sec:backend}.

Lediglich die Darstellung der Daten und die Nutzerinteraktion wird im Frontend abgewickelt.
Die Vorraussetzungen hierfür werden ausführlich im Kapitel \ref{sec:frontend} - \nameref{sec:frontend} beschrieben.

\subsubsection{Frontend}
\label{sec:frontend}
\subsubsection{Backend}
\label{sec:backend}

\subsection{Auswahl der Hardware}
\subsubsection{Clients}
\subsubsection{Server}

- Zusammenfassung der Anwendung, besteht aus einem Backend (Docker + MySQL + Python + Flask), und einem Frontend (umgesetzte WebApp, basierend auf TailwindCSS und HTMl, und JavaScript)
- Architektur der Anwendung: Bestehend aus einem Client / Server Modell --> Hardware beliebiger virtualisierungsfähiger Server, der in der Lage ist Container alias Docker / Kubernetes  zu hosten und einem Client, der keinerlei Vorraussetzungen als einen Webbrowser benötigt.
- Verteilung: Server der das Backend enthält und beliebige Clients, die sich zum Backend verbinden können.
- Auswahl des mobilen Endgerätes: Beliebiges mobiles Endgerät, da durch eine Serverseitige Anwendung wie Flask, und einer Client Umsetzung durch TailwindCSS keinerlei Vorraussetzungen benötigt werden, ist die App für alle mobilen Endgeräte, die einen Browser besitzen, der HTML5 und CSS3 beherrscht.