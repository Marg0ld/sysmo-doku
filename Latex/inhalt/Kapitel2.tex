\section{Architektur der Anwendung}
Beschreibung der Anwendung, Architektur der Anwendung hinsichtlich Hardware und Verteilung der Anwendung.
Dazu gehört auch die Auswahl des mobilen Endgerätes.

\subsection{Beschreibung der Anwendung}

\subsection{Aufbau der Anwendung}
\subsubsection{Allgemeiner Aufbau}
Die Anwendung ist wie in Abbildung \ref{fig:sysmo1} zu sehen aufgebaut. Diese teilt sich aus architektonischer Sicht in zwei grundlegende
Bereiche auf.

Auf der linken Seite ist der serverseitige Teil der Applikation zu sehen. Aufbauend auf der im folgenden Gliederungspunkt
beschriebenen Hardwareauswahl, setzt die Funktionalität der Serverapplikation auf einer beliebigen linuxbasierten Distribution auf. Dieses
bildet die Basis für das installieren einer Containerengine, welche für das Verwenden, Ausführen und Optimieren von sogennaten Containern 
verantwortlich ist. \striche{Ein Container ist ein Softwarepaket, das alles Wichtige zum Ausführen von Software enthält: 
Code, Laufzeit, Konfiguration und Systembibliotheken, damit das Programm auf jedem Hostsystem ausgeführt werden kann.}\onlinezitat{ITTALENTS} 
Grundlage für die Funktionalität beschriebener Container bildet Virtualisierung. Diese stellt der Engine Systemressourcen, in diesem Fall
in Form der Serverhardware zur Verfügung. Mittels dieser Funktionalität können Container betriebssystemunabhäniga ausgeführt, und je nach
Ressourcenbedarf betrieben werden. Aufbauend auf dieser technischen Grundlage werden 3 Container benötigt um die gewünschte Applikation 
betreiben zu können. 

\begin{enumerate}[leftmargin=*]
\item Datenbankcontainer: Dieser bildet die Grundlage der Anwendung, da mittels einer beliebigen Datenbank Produktdaten, deren Standort
      und weitere notwenige Daten verwaltet und gespeichert werden. Über eine Schnittstelle in Form einer \ac{API} können die restliche 
      Applikationscontainer auf diese zugreifen.

\item Webserver: Dieser stellt die grafische Benutzeroberfläche für den Client zur Verfügung. Das heißt, dieser überträgt Dokumente (.html, u.ä.)
      an den Client, welcher diese mittels Browser öffnet bzw. darstellt.

\item Software-Backend: Dieser Container bildet den Kern der serverseitigen Applikation. Alle Clientanfragen, Datenbankkomunikationen
      und Webserver Events werden von einem Programm innerhalb des Containers bearbeitet und gesteuert.      
\end{enumerate}

Zusammenfassend ist der serverseitige Aufbau der Anwendung die Grundlage für die Umsetzung der reinen, in der Zielstellung erläuterten 
Funktionalität. Für das Benutzen dieser, ist ein weiterer Architekturteil notwendig, welcher folgend erläutert wird.

Auf der rechten Seite der Abbildung \ref{fig:sysmo1} ist der clientseitige Teil der Applikation dargestellt. Unter der Maßgabe, 
die Anwendung möglichst vielen Nutzern zur Verfügung stellen zu können teilt sich das Schaubild in zwei Unterkategorien auf. 
Für das Nutzen der Anwendung im Supermarkt oder unterwegs, wird von seiten des Servers eine auf mobile Engeräte abgestimmte Website 
zur Verfügung gestellt. Diese kann betriebssystemunabhänig mittels des vorinstallierten Browser z.B. Safari, Chrome u.ä. aufgerufen werden.
Für das Verwenden von stationären Endgeräten wird ebenfals eine Website zur Verfügung gestellt, welche über einen Browser aufgerufen werden kann.
So ist die die Appliaktion auch für die Einkaufsplanung von zu Hause aus verwendbar. 

\bild[1.0]{sysmo1}{Theoretischer Aufbau der Anwendung}{fig:sysmo1}

Abschließend ist aus der architektonischen Betrachtung der Anwendung ersichtlich, dass diese möglichst plattformunahängig erstellt werden soll.
Innerhalb des Belegs soll nahezu die vollständige Abwicklung und Datenbereitstellung über das serverseitige contaier-basierte Backend erfolgen.
Eine ausführliche Beschreibung des Backends findet sich im Kapitel \ref{sec:backend} - \refname{sec:backend}.Lediglich die Darstellung der Daten 
und die Nutzerinteraktion wird im Frontend durcheführt.

\subsubsection{Frontend}
\label{sec:frontend}
\subsubsection{Backend}
\label{sec:backend}

\subsection{Auswahl der Hardware}
\subsubsection{Clients}
\subsubsection{Server}

- Zusammenfassung der Anwendung, besteht aus einem Backend (Docker + MySQL + Python + Flask), und einem Frontend
  (umgesetzte WebApp, basierend auf TailwindCSS und HTMl, und JavaScript)
- Architektur der Anwendung: Bestehend aus einem Client / Server Modell --> Hardware beliebiger virtualisierungsfähiger 
  Server, der in der Lage ist Container alias Docker / Kubernetes  zu hosten und einem Client, der keinerlei Vorraussetzungen 
  als einen Webbrowser benötigt.
- Verteilung: Server der das Backend enthält und beliebige Clients, die sich zum Backend verbinden können.
- Auswahl des mobilen Endgerätes: Beliebiges mobiles Endgerät, da durch eine Serverseitige Anwendung wie Flask, und einer 
  Client Umsetzung durch TailwindCSS keinerlei Vorraussetzungen benötigt werden, ist die App für alle mobilen Endgeräte, #
  die einen Browser besitzen, der HTML5 und CSS3 beherrscht.

  