\section{Bereich 2}

Aufgabe der einzelnen Komponenten in der verteilten Anwendung mit Spezifikation ihrer Schnittstelle hinsichtlich Datenaustausch (welche Daten in welchem Format).
Je nach Thema: Betriebssystem und eventuell notwendige Bibliotheken nicht vergessen.

- Einzelne Komponenten:
Server:
    Container-System --> Docker --> Compose
    Beliebige Datenbank, erklären, warum wir eine Datenbank brauchen, erklären, welche wir genommen haben: MySQL --> Compose dafür
    Webserver, alias Backend, erklären, was wir nehmen, warum Python, warum Flask --> Grundlegende Dockerfile, wie man einen eigenen Python Flask Container aufsetzt.
    Templates werden gerendert.
    Schnittstellen, zwischen Datenbank und Python, an der Stelle den Datenbankabrufcode reinballern.
    Datenformate ddie ausgetauscht werden, sollten wir grundlegend erklären --> Flowchart
    CDN --> Minio, um verschiedene Bilder zur Verfügung zu stellen, die man für die einzelnen Produkte braucht
Client:
    Benutzeroberfläche, die Daten von Flask abruft, dem Nutzer darstellt.
    An der Stelle mal gucken, wie viel wir dann noch brauchen, bzw. worauf man hier eingehen könnte.
    Darauf eingehen, dass auf Grund der Client -- Server-Struktur, dieser Teil nur minimale Anforderungen und Funktionalität besitzt.