\section{Einleitung und Aufgabenstellung}

\subsection{Themenauswahl}
Gewählt wurde Thema 3 - Suchen und Finden im Supermarkt.
Ein Supermarkt speichert in seiner Datenbank immer auch, in welchem Regal ein Produkt platziert wurde.
Ein Kunde kann bei der Auswahl eines Produktes in einer App sich anzeigen lassen, wo dieses Produkt im Markt zu finden ist.
Ebenfalls angezeigt werden Preis und vorrätige Menge.

\subsection{Aufgabenstellung}
\subsubsection{Bereich 1 - Architektur der Anwendung}
Beschreibung der Anwendung, Architektur der Anwendung hinsichtlich Hardware und Verteilung der Anwendung.
Dazu gehört auch die Auswahl des mobilen Endgerätes.

\subsubsection{Bereich 2 - Systemsoftware}
Aufgabe der einzelnen Komponenten in der verteilten Anwendung mit Spezifikation ihrer Schnittstelle hinsichtlich Datenaustausch (welche Daten in welchem Format).
Je nach Thema: Betriebssystem und eventuell notwendige Bibliotheken nicht vergessen.

\subsubsection{Bereich 3 - Umsetzung}
Auswahl einer Entwicklungsumgebung.
Skizze der Oberfläche der Anwendung (aus Sicht des Endanwenders).
Umsetzung einer ausgewählten Funktionalität in einer der Sprachen Python, C\#, Java oder Swift.

\subsection{Zielsetzung}
Bei einer derartigen Aufgabenstellung sind zahlreiche Umsetzungsmöglichkeiten gegeben. 
Unsere Gruppe hat sich dazu entschieden, eine webbasierte Lösung zu konzipieren.
Das Ziel ist es durch eine Client-Server-Architektur eine sehr hohe Flexibilität an mobilen Endgeräten zu ermöglichen.
Die Anforderungen auf der Serverseite sind bei diesem Ansatz deutlich höher, als auf der Clientseite, daher werden diese in den nachfolgenden Kapiteln ausführlich erläutert.
Generell möchten wir dabei auf die grundlegende Umsetzung einer solchen Lösung eingehen, genauere Implementierungsdetails werden nicht immer erklärt.
Alle verwendeten Codes, werden im Textteil teilweise und im Anhang vollständig eingefügt.
Der Beleg ist mit \LaTeX~nach Vorlage der HAWA\zitat{HAWA} und einer angepassten \LaTeX-Vorlage\zitat{Vorlage} erstellt.