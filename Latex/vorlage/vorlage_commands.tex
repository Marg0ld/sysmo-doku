%! Eigene Befehle zur erleichterten Nutzung
% Hilfsbefehle
\newcommand{\fontheightsvg}[1]{\includesvg[height=1.75ex, inkscapelatex=false]{#1}}
\newcommand{\dtfolder}{\fontheightsvg{vorlage/dirtree_folder}\hspace{0.1cm}}
\newcommand{\dtfile}{\fontheightsvg{vorlage/dirtree_file}\hspace{0.1cm}}

% Umgebungen u.Ä.
\newcommand{\fn}[1]{\footnote{\hspace{0.5em}#1}}
\newcommand{\bild}[4][1.0]{\begin{figure}[H]
  \centering
  \includegraphics[width=#1\columnwidth]{bilder/#2}
  \caption{#3}
  \label{fig:#4}
  \end{figure}}
\newcommand{\striche}[1]{\glqq #1\grqq{}}
\newcommand{\svg}[4][1.0]{\begin{figure}[H]
    \centering
    \includesvg[width=#1\columnwidth,inkscapelatex=false]{bilder/#2}
    \caption{#3}
    \label{fig:#4}
    \end{figure}}
\newcommand{\verzeichnis}[3]{\begin{figure}[H]
  % https://tex.stackexchange.com/a/99591/220899
  \renewcommand{\DTstyle}{\textrm\expandafter\raisebox{-0.7ex}}
  \centering
  \begin{varwidth}{\textwidth}
    \dirtree{#1}  
  \end{varwidth}
  \caption{#2}
  \label{fig:#3}
  \end{figure}}
\newcommand{\logisch}[1]{$``#1``$}
\newcommand{\vglink}[2]{\footnote{\hspace{0.5em}vgl.~\href{#1}{#1}~(#2)}}
\newcommand{\python}[1]{\mintinline{python}{#1}}

% Referenzierung
\newcommand{\uniliteref}[2]{\emph{\hyperref[{#2}]{#1 \ref{#2} - \nameref{#2}}}}
\newcommand{\unifullref}[2]{(\emph{\hyperref[{#2}]{siehe #1 \ref{#2} - \nameref{#2}}})}

% Anhänge
\newcommand{\litearef}[1]{\uniliteref{Anhang}{#1}}
\newcommand{\aref}[1]{\litearef{#1}} %Kompatibilität/Kurz
\newcommand{\fullaref}[1]{\unifullref{Anhang}{#1}}
% Abbildungen
\newcommand{\litebref}[1]{\uniliteref{Abbildung}{#1}}
\newcommand{\bref}[1]{\litearef{#1}} %Kompatibilität/Kurz
\newcommand{\fullbref}[1]{\unifullref{Abbildung}{#1}}
% Code
\newcommand{\litecref}[1]{\uniliteref{Code}{#1}}
\newcommand{\cref}[1]{\litearef{#1}} %Kompatibilität/Kurz
\newcommand{\fullcref}[1]{\unifullref{Code}{#1}}
% Formeln
\newcommand{\litecref}[1]{\uniliteref{Formel}{#1}}
\newcommand{\cref}[1]{\litearef{#1}} %Kompatibilität/Kurz
\newcommand{\fullcref}[1]{\unifullref{Formel}{#1}}
% Kapitel
\newcommand{\fullsref}[1]{\unifullref{Kapitel}{#1}}
\newcommand{\fullref}[1]{\fullsref{#1}} %Kompatibilität
\newcommand{\litesref}[1]{\uniliteref{Kapitel}{#1}}
\newcommand{\literef}[1]{\litesref{#1}} %Kompatibilität
\newcommand{\sref}[1]{\litesref{#1}} %kurz
% Tabellen
\newcommand{\litetref}[1]{\uniliteref{Tabelle}{#1}}
\newcommand{\tref}[1]{\litearef{#1}} %Kompatibilität/Kurz
\newcommand{\fulltref}[1]{\unifullref{Tabelle}{#1}}

