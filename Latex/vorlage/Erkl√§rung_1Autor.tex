%  Eidesstattliche Erklärung
%! Dies ist eine zur Nutzung mit LaTeX angepasste Version der in Anhang 6 der Hinweise zur Anfertigung
%! wissenschaftlicher Arbeiten an der Staatlichen Studienakademie Glauchau vorgegebenen Zustimmung
%! zur Plagiatsprüfung für Praxisbelege.
\cleardoublepage
\section{Ehrenwörtliche Erklärung}
    \vspace*{1cm}
    \begin{center}
        \huge\textbf{Ehrenwörtliche Erklärung}\\
    \end{center}
    \vspace*{1cm}
    \normalsize
    Ich erkläre hiermit ehrenwörtlich,

    \begin{enumerate}
        \vspace{1cm}
        \item dass ich meinen Praxisbeleg mit dem Thema:\\

        \textbf{\titel }\\

        ohne fremde Hilfe angefertigt habe,
        \item dass ich die Übernahme wörtlicher Zitate aus der Literatur sowie die\\
        Verwendung der Gedanken anderer Autoren an den entsprechenden\\
        Stellen innerhalb der Arbeit gekennzeichnet habe und
        \item dass ich meinen Praxisbeleg bei keiner anderen Prüfung vorgelegt habe.\\[1,5cm]
    \end{enumerate}
    Ich bin mir bewusst, dass eine falsche Erklärung rechtliche Folgen haben wird.\\[1,5cm]

    Glauchau, \abgabedatum \newline\noindent\rule{0.35\columnwidth}{0.4pt}\hspace{0.05\columnwidth}\rule{0.6\columnwidth}{0.4pt}\\
    Ort, Datum\hspace{0.27\columnwidth}Unterschrift

    \newpage
\section{Zustimmung Plagiatsprüfung}

    \vspace*{2mm}

    \begin{minipage}{0.5\columnwidth}
        \includesvg[width=\columnwidth]{vorlage/ba-gc-logo}
        % Alternativ mit PNG Logo, falls Inkscape nicht installiert werden, bzw. nicht der PATH-Variable hinzugefügt werden soll.
        %! Die SVG-Version sieht im Druck deutlich besser aus.
        %\includegraphics[width=\columnwidth]{vorlage/ba-gc-logo}
    \end{minipage}
    \begin{minipage}{0.45\columnwidth}
        \begin{flushright}
            {\small nach 4BA-F.219\\}
        \end{flushright}
    \end{minipage}
    \vspace*{2mm}

    \begin{center}
        \textbf{\huge{Erklärung zur Prüfung wissenschaftlicher Arbeiten}}
    \end{center}

    Die Bewertung wissenschaftlicher Arbeiten erfordert die Prüfung auf Plagiate. Die hierzu von der Staatlichen Studienakademie Glauchau eingesetzte Prüfungskommission nutzt sowohl eigene Software als auch diesbezügliche Leistungen von Drittanbietern. Dies erfolgt gemäß \href{https://www.revosax.sachsen.de/vorschrift/1672-Saechsisches-Datenschutzgesetz#p7}{§ 7 des Gesetzes zum Schutz der informationellen Selbstbestimmung im Freistaat Sachsen (Sächsisches Datenschutzgesetz - SächsDSG)} vom 25. August 2003 (Rechtsbereinigt mit Stand vom 31. Juli 2011) im Sinne einer Datenverarbeitung im Auftrag.

    Der Studierende bevollmächtigt die Mitglieder der Prüfungskommission hiermit zur Inanspruchnahme o. g. Dienste. In begründeten Ausnahmefällen kann der Datenschutzbeauftragte der Berufsakademie Sachsen sowohl vom Verfasser der wissenschaftlichen Arbeit als auch von der Prüfungskommission in den Entscheidungsprozess einbezogen werden.

    \arrayrulewidth=0.5pt
    \begin{table}[H]
        \centering
        \begin{tabularx}{\columnwidth}{|p{3.2cm}|X|}
            \hline
            Name:             & \autoreins\\
            \hline
            Matrikelnummer:   & \matnumeins\\
            \hline
            Studiengang:      & \studiengang\\
            \hline
            Titel der Arbeit: & \titel\\
            \hline
            Datum:            & \abgabedatum\\
            \hline
            Unterschrift:     & \\
                              & \\
            \hline
        \end{tabularx}
    \end{table}

    \vfill
