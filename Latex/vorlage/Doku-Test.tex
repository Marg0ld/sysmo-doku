%Alle Kapitel beginnen mit der Hauptüberschrift
\section{Einleitung}
Diese PDF wurde mit der Vorlage erstellt, um die Funktion und Formatierung dieser zu zeigen.

Die Vorlage\onlinezitat{SCZYRBA2020} ist ein Gemeinschaftsprojekt im Rahmen unseres Studiums.
Der Docker-Container\onlinezitat{HILLE2021} gehört dazu.
Die Vorlage richtet sich weitestgehend nach dem Dokument \ac{HAWA}\onlinezitat{HAWA} der \href{https://www.ba-glauchau.de/}{Staatlichen Studienakademie Glauchau}.

Weitere Hinweise befinden sich in der README.md oder im \href{https://github.com/DSczyrba/Vorlage-Latex/wiki}{Wiki}.
Eine ausführliche Dokumentation zu diesem Dokument wird folgen.

\section{Beispiele}
\label{sec:beispiele}
\subsection{Pixelgrafiken}
Siehe Quelltext.\fn{Dort wird das Kommando \striche{bild} aufgeführt. In dieser Vorlage sind allerdings keine Bilder (im vorgesehenen Ordner) enthalten.}
%\bild[0.5]{dateiname-ohne-endung}{Text zu einem Bild}{label1}
\subsection{Vektorgrafiken}
Siehe Quelltext.
%\svg[0.5]{dateiname-ohne-endung}{Text zu einer \ac{SVG}-Datei}{label2}
\subsubsection{Programmcode}
\begin{code}[H]
    \begin{minted}{python}
        import asdf
        from foo import bar
        
        def yolo():
            return "glhf"
        
        class Wooo(Foo):
            def __init__(self, boo):
                self.doo = boo
                moo = 1 + 2 +3
    \end{minted}
    \caption{Beispielcode}
    \label{code:example}
\end{code}
An dieser Stelle wird noch ein einzeiliger Code eingefügt, der keine Zeilennummerierung erhalten soll.
Dies kann genutzt werden, wenn man nur kurz auf eine Funktion eingehen möchte und diese nicht im Quellcodeverzeichnis erscheinen soll.
\mint{python}|print("Hallo Vorlage")|

Ein weiterer Code um zu schauen, ob danach die Zeilennummerierung wieder funktioniert.

\begin{code}[H]
  \begin{minted}{python}
      class Wooo(Foo):
          def __init__(self, boo):
              self.doo = boo
              moo = 1 + 2 +3
  \end{minted}
  \caption{Beispielcode}
  \label{code:example2}
\end{code}

\subsection{Ordnerstruktur}
    \verzeichnis{%
          .1 \dtfolder Vorlage-Latex.
            .2 \dtfile HINWEISE.md.
            .2 \dtfile LICENSE.
            .2 \dtfile README.md.
            .2 \dtfile sortieren.py.
            .2 \dtfolder Latex.
              .3 \dtfolder bilder.
                .4 \dtfile firmenlogo.svg.
              .3 \dtfolder inhalt.
                .4 \dtfile Abkürzungen.tex.
                .4 \dtfile Anhang.tex.
                .4 \dtfile Kapitel1.tex.
              .3 \dtfolder light.
                .4 \dtfile main.tex.
                .4 \dtfile README.md.
                .4 \dtfile vorlage\_light.tex.
              .3 \dtfile literatur.bib.
              .3 \dtfile main.tex.
              .3 \dtfile metadaten.sty.
              .3 \dtfolder vorlage.
                .4 \dtfile Abstract\_Freigabeerklärung\_Bachelorthesis.tex.
                .4 \dtfile ba-gc-logo.png.
                .4 \dtfile ba-gc-logo.svg.
                .4 \dtfile dirtree\_file.svg.
                .4 \dtfile dirtree\_folder.svg.
                .4 \dtfile Erklärung\_3Autoren.tex.
                .4 \dtfile Erklärung\_4Autoren.tex.
                .4 \dtfile Erklärung\_Praxisbeleg.tex.
                .4 \dtfile Titelseite\_3Autoren.tex.
                .4 \dtfile Titelseite\_4Autoren.tex.
                .4 \dtfile Titelseite\_Praxisbeleg.tex.
                .4 \dtfile vorlage.tex.
    }{Ein Verzeichnis-Baum}{beispielbaum}

\section{Test-/Dokutabelle}
Diese Tabelle dient hauptsächlich zum Testen der einzelnen Kommandos, sowie als minimales Beispiel für eine \emph{tabularx}-Tabelle:
\begin{table}[H]
\begin{tabularx}{\columnwidth}{|p{3cm}|X|p{.2\columnwidth}|}
\hline
Gegenstand & Beispiel & Anmerkungen \\
\hline
Kurzer Verweis auf Kapitel & \literef{sec:beispiele} & \\
\hline
Kurzer Verweis auf Anhang & \litearef{sec:anhang1} & Alias: \emph{\textbackslash aref} \\
\hline
Kurzer Verweis auf Abbildung & \litebref{beispielbaum} & Alias: \emph{\textbackslash bref} \\
\hline
Kurzer Verweis auf Tabelle & \litetref{beispieltabelle} & Alias: \emph{\textbackslash tref} \\
\hline
Langer Verweis auf Kapitel & \fullref{sec:beispiele} & \\
\hline
Langer Verweis auf Anhang & \fullaref{sec:anhang1} &  \\
\hline
Langer Verweis auf Abbildung & \fullbref{beispielbaum} &  \\
\hline
Langer Verweis auf Tabelle & \fulltref{beispieltabelle} &  \\
\hline
\multicolumn{2}{|c|}{verbundene Spalten mit zentriertem Text} & \emph{\textbackslash multicolumn} \\
\hline
Zeile 1 & \multirow{2}{\hsize}{verbundene Zeilen inklusive automatischer Zeilenumbrüche} & \emph{\textbackslash multirow} \\
\cline{1-1}\cline{3-3}
Zeile 2 & & mit \emph{\textbackslash hsize}\\
\hline
\end{tabularx}
\caption{Beispieltabelle}
\label{beispieltabelle}
\end{table}
